%---------------------------------------------
% This document is for pdflatex
%---------------------------------------------
\documentclass[11pt]{article}

\usepackage{amsmath,amsfonts,amssymb,graphicx,setspace,authblk}
\usepackage{float}
\usepackage[running]{lineno}
\usepackage[vmargin=1in,hmargin=1in]{geometry}

\usepackage[authoryear]{natbib}

\graphicspath{ {./../genetic-diversity/figures/} }

\usepackage{enumitem}
\setlist{topsep=.125em,itemsep=-0.15em,leftmargin=0.75cm}

\usepackage{gensymb}

\usepackage[compact]{titlesec} 

\usepackage{bm,mathrsfs}

\usepackage{ifpdf}
\ifpdf
\DeclareGraphicsExtensions{.pdf,.png,.jpg}
\usepackage{epstopdf}
\else
\DeclareGraphicsExtensions{.eps}
\fi

\renewcommand{\floatpagefraction}{0.98}
\renewcommand{\topfraction}{0.99}
\renewcommand{\textfraction}{0.05}

\clubpenalty = 10000
\widowpenalty = 10000

%%%%%%%%%%%%%%%%%%%%%%%%%%%%%%%%%%%%%%%%%%%%% 
%%% Just for commenting
%%%%%%%%%%%%%%%%%%%%%%%%%%%%%%%%%%%%%%%%%%%%
\usepackage[usenames]{color}
\newcommand{\new}{\textcolor{red}}
\newcommand{\spe}{\textcolor{blue}}
\newcommand{\comment}{\textcolor{black}}

\newcommand{\be}{\begin{equation}}
\newcommand{\ee}{\end{equation}}
\newcommand{\ba}{\begin{equation} \begin{aligned}}
\newcommand{\ea}{\end{aligned} \end{equation}}

\def\X{\mathbf{X}}

\floatstyle{boxed}
\newfloat{Box}{tbph}{box}

\title{Integrating spatial and temporal information to make ecological forecasts }

\author[1]{Peter B. Adler\thanks{Corresponding author. Department of Wildland Resources and the Ecology Center, Utah State University, Logan, Utah Email: peter.adler@usu.edu}}
\author[2]{Ethan White}
\author[3]{others?}
\affil[1]{Department of Wildland Resources and the Ecology Center, Utah State University, Logan, Utah}
\affil[2]{some shitty Florida joint}
\affil[3]{??}

\renewcommand\Authands{ and }

% \date{Last compile: \today} 

\sloppy

\renewcommand{\baselinestretch}{1.25}

\begin{document}

\maketitle

\linenumbers

\section*{Abstract}

stuff

\textbf{\large{Keywords:}} Coexistence, competition, integral projection model, removal experiment, sagebrush steppe. 

\section*{Introduction}

Cheer lead for forecasting yada yada 

Strengths and weaknesses of spatial approach

Strengths of weaknesses of temporal approach

Let's take the best of both worlds! Combine with weights that change over time. We might learn something interesting from the rate at which the weights change.

\section*{Eco-evo case study}

Consider an annual plant population in which fecundity is temperature dependent, and different genotypes have different temperature optima (Fig. \ref{fig:RxnNorms}).

\begin{figure}[tbp]
\centering
\includegraphics[width=0.6 \textwidth]{reactionnorms.png}
\caption{Reaction norms of the three genotypes.}
\label{fig:RxnNorms}
\end{figure}

Genotype frequencies will shift across a gradient of mean annual temperature: cold sites will be dominated by the cold-loving homozygous genotype, warm sites will be dominated by the heat-loving homozygous genotype, and intermediate sites will be dominated by the heterozygous genotype (Fig. \ref{fig:SpatialPattern}).

\begin{figure}[tbp]
\centering
\includegraphics[width=0.75 \textwidth]{spatial_pattern.png}
\caption{The spatial pattern of individual genotypes and total population abundances at sites arrayed across a gradient of mean annual temperature. The dashed line shows predictions from the ``spatial model."}
\label{fig:SpatialPattern}
\end{figure}

This spatial pattern is the outcome of steady-state conditions. At any one site, the population's short-term response to temperature will be determined by the dominant genotype's reaction norm. For example, at a cold site dominated by the cold-loving homozygous genotype, a warmer than normal year would cause a decrease in population size, even though the heat-loving homozygote might perform optimally at that temperature. If warmer than normal conditions persist for many years, then genotype frequencies will shift, and the heat-loving homozygote will compensate for the decreases of the cold-loving genotype. 


%\newpage
%\renewcommand{\refname}{Literature cited}
%\bibliographystyle{Ecology}
%\bibliography{RemovalRefs}


%~~~~~~~~~~~~~~~~~~~~~~~~~~~~~~~~~~~~~~~~~~~~~~~~~~~~~~~~~~~~~~~~~~~~~~~~~~~~~
% APPENDICES !
%~~~~~~~~~~~~~~~~~~~~~~~~~~~~~~~~~~~~~~~~~~~~~~~~~~~~~~~~~~~~~~~~~~~~~~~~~~~~~

%\clearpage 
%\newpage 
%
%\setcounter{page}{1}
%\setcounter{equation}{0}
%\setcounter{figure}{0}
%\setcounter{section}{0}
%\setcounter{table}{0}
%
%\centerline{\Large \textbf{Appendices}}
%\centerline{Adler et al., ``Weak interspecific interactions''} 
%
%\vspace{0.4in} 
%
%\renewcommand{\theequation}{A-\arabic{equation}}
%\renewcommand{\thetable}{A-\arabic{table}}
%\renewcommand{\thefigure}{A-\arabic{figure}}
%\renewcommand{\thesection}{\Alph{section}}

\end{document}

